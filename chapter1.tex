\section*{Chapter 1}
\subsection*{Preliminaries: Set theory and categories}
\setcounter{subsection}{1}

% Problem 1.1
\begin{problem}
  Locate a discussion of Russel's paradox, and understand it.
\end{problem}
\begin{solution}
  Russell's Paradox arises within naive set theory when considering the set
  $R$ of all sets that are not members of themselves. If $R \in R$, then by
  definition of $R$, $R \notin R$, a contradiction. If $R \notin R$, then $R$
  does not contain itself, hence $R$ must be a member of $R$,
  a contradiction.
\end{solution}

% Problem 1.2
\begin{problem}
  $\rhd$ Prove that if $\sim$ is an equivalence relation on a set $S$, then
  the corresponding family $S/\sim$ defined in \S1.5 is indeed a
  partition of $S$; that is, its elements are nonempty, disjoint, and their
  union is $S$.
\end{problem}
\begin{solution}
  Consider the family of equivalence classes w.r.t. $\sim$ over $S$:
  %
  \[ S/\sim = \set{[a]_{\sim} \mid a\in S} \]
  %
  Let $[a]_{\sim}\in S/\sim$. Since $\sim$ is an equivalence relation,
  by reflexivity we have $a\sim a$, so $[a]_{\sim}$ is nonempty. Now, suppose
  $a$ and $b$ are arbitrary elements in $S$ such that $a\not\sim b$. Suppose
  that there is an $x\in [a]_{\sim}\cap[b]_{\sim}$. This means that $x\sim a$
  and $x\sim b$. By transitivity, we get that $a\sim b$; contradiction.
  Hence the $[a]_{\sim}$ are disjoint. Finally, let $x\in S$. Then
  $x\in[x]_{\sim}\in S/\sim$. This means that
  %
  \[ \bigcup_{[a]_{\sim} \in \mathscr{P}_{\sim}} [a]_{\sim} \supseteq S, \]
  %
  that is, the union of the elements of $S/\sim$ contains $S$.
  On the other hand, $\forall x \in [a]_\sim, x \in S$ by the definition
  of $[a]_\sim$, and therefore:
  %
  \[ \bigcup_{[a]_{\sim} \in \mathscr{P}_{\sim}} [a]_{\sim} \subseteq S, \]
  %
  And we conclude that the two sets must be equal.
\end{solution}

% Problem 1.3
\begin{problem}
  $\rhd$ Given a partition $\mathscr{P}$ on a set $S$, show how to define a
  relation $\sim$ such that $\mathscr{P} = \mathscr{P}_{\sim}$. [\S1.5]
\end{problem}

\begin{solution}
We think of $\mathscr{P}$ as an indexed set of subsets of $S$. Then we define
$a \sim b$ if and only if $a,b \in \mathscr{P}_k$. Note the following.

\begin{itemize}
    \item $a \in \mathscr{P}_k \implies a \sim a$
    \item $a \sim b \implies a,b \in \mathscr{P}_k \implies b \sim a$
    \item $a \sim b$ and $b \sim a \implies a,b,c \in \mathscr{P}_k \implies a \sim c$
\end{itemize}

So $\sim$ is an equivalence relation and we can speak of its equivalence
classes. If we let $a \in \mathscr{P}_k$, then it follows that $[a] =
\mathscr{P}_k$ because $a \sim b \iff a,b \in \mathscr{P}_k$.
\end{solution}


% Problem 1.4
\begin{problem}
  How many different equivalence relations can be defined on the set
  $\{1,2,3\}?$
\end{problem}
\begin{solution}
  From the definition of an equivalence relation and the solution to problem
  \textbf{I.1.3}, we can see that an equivalence relation on $S$ is equivalent
  to a partition of $S$. Thus the number of equivalence relations on $S$ is
  equal to the number of partitions of $S$. Since $\{1,2,3\}$ is small we can determine this by hand:
  %
  \[ \mathscr{P}_0 = \set{\{1,2,3\}} \]
  \[ \mathscr{P}_1 = \set{\{1\},\{2\},\{3\}\}} \]
  \[ \mathscr{P}_2 = \set{\{1,2\},\{3\}} \]
  \[ \mathscr{P}_3 = \set{\{1\},\{2,3\}} \]
  \[ \mathscr{P}_4 = \set{\{1,3\},\{2\}} \]
  %
  Thus there can be only $5$ equivalence relations defined on $\{1,2,3\}$.
\end{solution}

% Problem 1.5
\begin{problem}
  Give an example of a relation that is reflexive and symmetric but not
  transitive. What happens if you attempt to use this relation to define a
  partition on the set? (Hint: Thinking about the second question will help
  you answer the first one.)
\end{problem}
\begin{solution}
  For $a,b\in \mathbb{Z}$, define $a\diamond b$ to be true if and only if
  $\abs{a-b} \leq 1$. It is reflexive, since $a\diamond a = \abs{a-a} = 0 \leq
  1$ for any $a\in \mathbb{Z}$, and it is symmetric since $a\diamond b =
  \abs{a-b} = \abs{b-a} = b\diamond a$ for any $a,b\in \mathbb{Z}$. However,
  it is not transitive. Take for example $a=0, b=1, c=2$.  Then we have
  $\abs{a-b} = 1\leq 1$, and $\abs{b-c} = 1\leq 1$, but $\abs{a-c} = 2 > 1$;
  so $a\diamond b$ and $b\diamond c$, but not $a\diamond c$.

  When we try to build a partition of $\mathbb{Z}$ using $\diamond$, we get
  "equivalence classes" that are not disjoint. For example, $[2]_{\diamond} =
  \{1,2,3\}$, but $[3]_{\diamond} = \{2,3,4\}$. Hence $\mathscr{P}_{\diamond}$
  is not a partition of $\mathbb{Z}$.
\end{solution}

% Problem 1.6
\begin{problem}
  Define a relation $\sim$ on the set $\mathbb{R}$ of real numbers by
  setting $a\sim b\iff b-a\in\mathbb{Z}$. Prove that this is an equivalence
  relation, and find a `compelling' description for $\mathbb{R}\,/\!\sim$.
  Do the same for the relation $\approx$ on the plane $\mathbb{R}\times
  \mathbb{R}$ by declaring $(a_1,b_1)\approx(a_2,b_2)\iff
  b_1-a_1\in\mathbb{Z}$ and $b_2-a_2\in\mathbb{Z}$. [\S II.8.1, II.8.10]
\end{problem}
\begin{solution}
  Suppose $a,b,c\in\mathbb{R}$. We have that $a-a=0\in\mathbb{Z}$, so $\sim$
  is reflexive. If $a\sim b$, then $b-a=k$ for some $k\in\mathbb{Z}$, so
  $a-b=-k\in\mathbb{Z}$, hence $b\sim a$. So $\sim$ is symmetric. Now, suppose that $a\sim b$ and $b\sim c$, in particular that $b-a=k\in\mathbb{Z}$ and
  $c-b=l\in\mathbb{Z}$. Then $c-a=(c-b) + (b-a) = l+k\in\mathbb{Z}$, so
  $a\sim c$. So $\sim$ is transitive.

  Doing the same component-wise on for the relation $\approx$ on the plane
  $\mathbb{R}\times \mathbb{R}$ is trivial.

  An equivalence class $[a]_{\sim}\in\mathbb{R}\,/\!\sim$ is the set of
  integers $\mathbb{Z}$ transposed by some real number $\epsilon\in[0,1)$. That is, for every set $X\in\mathbb{R}\,/\!\sim$, there is a real number
  $\epsilon\in[0,1)$ such that every $x\in X$ is of the form $k+\epsilon$ for some integer $k$.

  If we map each $x\in\mathbb{R}$ onto a circle using $\varphi=2\pi x$, then
  each equivalence class $[a]_{\sim}\in\mathbb{R}\,/\!\sim$ will correspond to
  a single point on its circumference.

  The interpretation of $\approx$ is similar to $\sim$. An equivalence class
  $X\in\mathbb{R}\times\mathbb{R}\,/\approx$ is just the 2-dimensional integer lattice $\mathbb{Z}\times\mathbb{Z}$ transposed by some pair of values $(\epsilon_1,\epsilon_2)\in[0,1)\times[0,1)$. $X$ can be visualized as a point
  on a torus.
\end{solution}

\subsection{Functions between sets}


% Problem 2.1
\begin{problem}
  How many different bijections are there between a set with $n$ elements
  and itself?
\end{problem}
\begin{solution}
  A function $f:S\to S$ is a graph $\Gamma_f\subseteq S\times S$. Since $f$ is
  bijective, then for all $y\in S$ there exists a unique $x\in S$ such that
  $(x,y)\in\Gamma_f$. We can see that $\abs{\Gamma_f} = n$. Since each $x$ must
  be unique, all the elements $x\in S$ must be present in the first component
  of exactly one pair in $\Gamma_f$. Furthermore, if we order the elements
  $(x,y)$ in $\Gamma_f$ by the first component, we can see that $\Gamma_f$ is just a permutation on the $n$ elements in $S$. For example, for $S=\{1,2,3\}$ one such $\Gamma_f$ is:
  %
  \[ \set{ (1,3), (2,2), (3,1) } \]
  %
  Since $\abs{S} = n$, the number of permutations of $S$ is $n!$. Hence there
  can be $n!$ different bijections between $S$ and itself.
\end{solution}

% Problem 2.2
\begin{problem}
  $\rhd$ Prove statement (2) in Proposition 2.1. You may assume that given a
  family of disjoint subsets of a set, there is a way to choose one element
  in each member of the family. [\S2.5, V3.3]

  \begin{quote}
    \textbf{Proposition 2.1.} Assume $A\neq\emptyset$, and let $f:A\to B$ be
    a function. Then \\
      (1) $f$ has a left-inverse if and only if $f$ is injective; and \\
      (2) $f$ has a right-inverse if and only if $f$ is surjective.
  \end{quote}
\end{problem}
\begin{solution}
  Let $A\neq\emptyset$ and suppose $f:A\to B$ is a function.

  ($\implies$) Suppose there exists a function $g$ that is a right-inverse of
  $f$. Then $f\circ g = \id_B$. Let $b\in B$. We have that $f(g(b)) = b$, so
  there exists an $a = g(b)$ such that $f(a) = b$. Hence $f$ is surjective.

  ($\impliedby$) Suppose that $f$ is surjective. We want to construct a function
  $g:B\to A$ such that $f(g(a)) = a$ for all $a\in A$. Since $f$ is surjective,
  for all $b\in B$ there is an $a\in A$ such that $f(a) = b$. For each $b\in B$
  construct a set $\Lambda_b$ of such pairs:
  %
  \[ \Lambda_b = \set{ (a,b) \mid a \in A, f(a) = b } \]
  %
  Note that $\Lambda_b$ is non-empty for all $b\in B$. So that we can choose one
  pair $(a,b)$ ($a$ not necessarily unique) from each set in $\Lambda =
  \set{\Lambda_b\mid b\in B}$ to define $g:B\to A$:
  %
  \[ g(b) = a, \text{ where $a$ is in some $(a,b)\in\Lambda_b$} \]
  %
  Now, $g$ is a right-inverse of $f$. To show this, let $b\in B$. Since $f$ in
  surjective, $g$ has been defined such that when $a=g(b)$, $f(a)=b$, so we get
  that $f(g(b)) = (f\circ g)(b) = b$, thus $g$ is a right-inverse of $f$.

  % TODO: injectivity and left infverses
\end{solution}

% Problem 2.3
\begin{problem}
  Prove that the inverse of a bijection is a bijection and that the
  composition of two bijections is a bijection.
\end{problem}

% Problem 2.4
\begin{problem}
  $\rhd$ Prove that `isomorphism' is an equivalence relation (on any set
  of sets.) [\S4.1]
\end{problem}

% Problem 2.5
\begin{problem}
  $\rhd$ Formulate a notion of \textit{epimorphism}, in the style
  of the notion of \textit{monomorphism} seen in \S 2.6, and prove a result
  analogous to Proposition 2.3, for epimorphisms and surjections.
\end{problem}

% Problem 2.6
\begin{problem}
  With notation as in Example 2.4, explain how any function $f:A\to B$
  determines a section of $\pi_A$.
\end{problem}

% Problem 2.7
\begin{problem}
  Let $f:A\to B$ be any function. Prove that the graph $\Gamma_f$ of $f$ is
  isomorphic to $A$.
\end{problem}

% Problem 2.8.
\begin{problem}
  Describe as explicitly as you can all terms in the canonical decomposition
  (cf. \S2.8) of the function $\mathbb{R}\to\mathbb{C}$ defined by $r\mapsto
  e^{2\pi ir}$. (This exercise matches one previously. Which one?)
\end{problem}

% Problem 2.9
\begin{problem}
  $\rhd$ Show that if $A'\cong A''$ and $B'\cong B''$, and further
  $A'\cap B'=\emptyset$ and $A''\cap B''=\emptyset$, then
  $A'\cup B'\cong A''\cup B''$. Conclude that the operation $A\amalg B$
  is well-defined up to \textit{isomorphism} (cf. \S2.9) [\S2.9, 5.7]
\end{problem}

% Problem 2.10
\begin{problem}
  $\rhd$ Show that if $A$ and $B$ are finite sets, then $\abs{B^A} =
  \abs{B}^{\abs{A}}$. [\S2.1, 2.11, I.4.1]
\end{problem}

% Problem 2.11
\begin{problem}
  $\rhd$ In view of Exercise 2.10, it is not unreasonable to use $2^A$ to
  denote the set of functions from an arbitrary set $A$ to a set with $2$
  elements (say $\{0,1\}$). Prove that there is a bijection between $2^A$
  and the \textit{power set} of $A$ (cf. \S1.2). [\S1.2, III.2.3]
\end{problem}

\subsection{Category theory}


% Problem 3.1
\begin{problem}
  $\rhd$ Let $\mathsf{C}$ be a category. Consider a structure
  $\mathsf{C}^{op}$ with
  \begin{enumerate}
    \item $\Obj(\mathsf{C}^{op}) = \Obj(\mathsf{C})$
    \item For $A, B$ objects of $\mathsf{C}^{op}$ (hence objects of
    $\mathsf{C}$), $\Hom_{\mathsf{C}^{op}}(A, B) := \Hom_{\mathsf{C}}(B, A)$.
  \end{enumerate}
  Show how to make this into a category (that is, define composition of
  morphisms in $\mathsf{C}^{op}$ and verify the properties listed in \S3.1).

  Intuitively, the `opposite' category $\mathsf{C}^{op}$ is simply obtained
  by `reversing all the arrows' in $\mathsf{C}$. [5.1, \S III.1.1, \S IX.1.2,
  IX.1.10]
\end{problem}

% Problem 3.2
\begin{problem}
  If $A$ is a finite set, how large is $\mathrm{End}_{\mathsf{Set}}(A)$?
\end{problem}

% Problem 3.3
\begin{problem}
  $\rhd$ Formulate precisely what it means to say that $1_a$ is an identity
  with respect to composition in Example 3.3, and prove this assertion.
  [\S3.2]
\end{problem}

% Problem 3.4
\begin{problem}
  Can we define a category in the style of Example 3.3 using the relation
  $<$ on the set $\mathbb{Z}$?
\end{problem}

% Problem 3.5
\begin{problem}
  $\rhd$ Explain in what sense Example 3.4 is an instance of the categories
  considered in Example 3.3. [\S 3.2]
\end{problem}

% Problem 3.6
\begin{problem}
  $\rhd$ (Assuming some familiarity with linear algebra.) Define a category
  $\mathsf{V}$ by taking $\Obj(\mathsf{V}) = \mathbb{N}$ and letting
  $\Hom_{\mathsf{V}}(m,n) = $ the set of $m\times n$ matrices with real
  entries, for all $m,n\in\mathbb{N}$. (We will leave the reader the task of
  making sense of a matrix with 0 rows or columns.) Use product of matrices
  to define composition. Does this category `feel' familiar?
  [\S VI.2.1, \S VIII.1.3]
\end{problem}

% Problem 3.7
\begin{problem}
  $\rhd$ Define carefully the objects and morphisms in Example 3.7, and draw
  the diagram corresponding to compositon. [\S 3.2]
\end{problem}

% Problem 3.8
\begin{problem}
  \def \C {\mathsf{C}}
  \def \Cp {\C'}

  $\rhd$ A \textit{subcategory} $\Cp$ of a category $\C$ consists of a
  collection of objects of $\C$, with morphisms $\Hom_\Cp(A,B) \subseteq
  \Hom_\C(A,B)$ for all objects $A,B\in\Obj(\Cp)$, such that identities and
  compositions in $\C$ make $\Cp$ into a category. A subcategory $\Cp$ is
  \textit{full} if $\Hom_\Cp(A,B) = \Hom_\C(A,B)$ for all $A,B\in\Obj(\Cp)$.
  Construct a category of \textit{infinite sets} and explain how it may be
  viewed as a full subcategory of $\mathsf{Set}$. [4.4,\S VI.1.1,
  \S VIII.1.3]
\end{problem}

% Problem 3.9
\begin{problem}
  \def \Set {\mathsf{Set}}
  \def \MSet {\mathsf{MSet}}

  $\rhd$ An alternative to the notion of \textit{multiset} introduced in
  \S2.2 is obtained by considering sets endowed with equivalence relations;
  equivalent elements are taken to be multiple instance of elements `of the
  same kind'. Define a notion of morphism between such enhanced sets,
  obtaining a category $\MSet$ containing (a `copy' of) $\Set$ as a full
  subcategory. (There may be more than one reasonable way to do this!
  This is intentionally an open-ended exercise.) Which objects in $\MSet$
  determine ordinary multisets as defined in \S2.2 and how? Spell out what
  a morphism of multisets would be from this point of view. (There are
  several natural motions of morphisms of multisets. Try to define morphisms
  in $\MSet$ so that the notion you obtain for ordinary multisets captures
  your intuitive understanding of these objects.) [\S2.2, \S3.2, 4.5]
\end{problem}

% Problem 3.10
\begin{problem}
  \def \C {\mathsf{C}}
  \def \Set {\mathsf{Set}}

  Since the objects of a category $\C$ are not (necessarily) sets, it is not
  clear how to make sense of a notion of `subobject' in general. In some
  situations it \textit{does} make sense to talk about subobjects, and the
  subobjects of any given object $A$ in $\C$ are in one-to-one correspondence
  with the morphisms $A\to\Omega$ for a fixed, special object $\Omega$ of
  $\C$, called a \textit{subobject classifier}. Show that $\Set$ has
  a subobject classifier.
\end{problem}

% Problem 3.11
\begin{problem}
  \def \C {\mathsf{C}}

  $\rhd$ Draw the relevant diagrams and define composition and identities for
  the category $\C^{A,B}$ mentioned in Example 3.9. Do the same for the
  category $\C^{\alpha,\beta}$ mentioned in Example 3.10. [\S5.5, 5.12]
\end{problem}

\subsection{Morphisms}


% Problem 4.1
\begin{problem}
  $\rhd$ Composition is defined for \textit{two} morphisms. If more than two
  morphisms are given, e.g.,
  %
  \[ A \xrightarrow{f} B \xrightarrow{g} C \xrightarrow{h} D \xrightarrow{i} E \]
  %
  then one may compose them in several ways, for example,
  %
  \[ (ih)(gf),\,\,\,\,(i(hg))f,\,\,\,\,i((hg)f),\,\,\,\,\text{etc.}\]
  %
  so that at every step one is only composing two morphisms. Prove that the
  result of any such nested composition is independent of the placement of
  the parentheses.
\end{problem}

% Problem 4.2
\begin{problem}
  $\rhd$ In Example 3.3 we have seen how to construct a category from a set
  endowed with a relation, provided this latter is reflexive and transitive.
  For what types of relations is the corresponding category a groupoid (c.f.
  Example 4.6)? [\S 4.1]
\end{problem}

% Problem 4.3
\begin{problem}
  \def \C {\mathsf{C}}
  Let $A$, $B$ be objects of a category $\C$, and let $f \in \Hom_\C(A, B)$
  be a morphism.
  \begin{itemize}
    \item Prove that if $f$ has a right-inverse, then $f$ is an epimorphism.
    \item Show that the converse does not hold, by giving an explicit example
    of a category and an epimorphism without a right-inverse.
  \end{itemize}
\end{problem}

% Problem 4.4
\begin{problem}
  \def \C {\mathsf{C}}
  \def \mono {\mathsf{mono}}
  \def \nonmono {\mathsf{nonmono}}

  Prove that the composition of two monomorphisms is a monomorphism. Deduce
  that one can define a subcategory $\C_\mono$ of a category $\C$ by taking
  the same objects as in $\C$ and defining $\Hom_{\C_\mono}(A, B)$ to be the
  subset of $\Hom_\C(A, B)$ consisting of monomorphisms, for all objects $A,
  B$. (Cf. Exercise 3.8; of course, in general $\C_\mono$ is not full in
  $C$.) Do the same for epimorphisms. Can you define a subcategory
  $\C_\nonmono$ of $\C$ by restricting to morphisms that are not
  monomorphisms?
\end{problem}

% Problem 4.5
\begin{problem}
  \def \MSet {\mathsf{MSet}}
  Give a concrete description of monomorphisms and epimorphisms in the
  category $\MSet$ you constructed in Exercise 3.9. (Your answer will depend
  on the notion of morphism you defined in that exercise!)
\end{problem}

\subsection{Universal Properties}

% Problem 5.1
\begin{problem}
  \def \C {\mathsf{C}}
  \def \Cop {{\mathsf{C}^{op}}}
  Prove that a final object in a category $\C$ is initial in the opposite
  category $\Cop$ (cf. Exercise 3.1).
\end{problem}

% Problem 5.2
\begin{problem}
  \def \Set {\mathsf{Set}}
  $\rhd$ Prove that $\emptyset$ is the unique initial object in $\Set$.
  [\S 5.1].
\end{problem}

% Problem 5.3
\begin{problem}
  $\rhd$ Prove that final objects are unique up to isomorphism. [\S 5.1]
\end{problem}

% Problem 5.4
\begin{problem}
  What are initial and final objects in the category of `pointed sets'
  (Example 3.8)? Are they unique?
\end{problem}

% Problem 5.5
\begin{problem}
  Consider the category corresponding to endowing (as in Example 3.3) the set
  $\mathbb{Z}^+$ of positive integers with the divisibility relation. Thus
  there is exactly one morphism $d\to m$ in this category if and only if $d$
  divides $m$ without remainder; there is no morphism between $d$ and $m$
  otherwise. Show that this category has products and coproducts. What are
  their `conventional' names? [\S VII.5.1]
\end{problem}

% Problem 5.5
\begin{problem}
  What are the final objects in the category considered in \S5.3? [\S5.3]
\end{problem}

% Problem 5.6
\begin{problem}
  Redo Exercise 2.9, this time using Proposition 5.4.
\end{problem}

% Problem 5.7
\begin{problem}
  \def \C {\mathsf{C}}
  Show that in every category $\C$ the products $A\times B$ and $B\times A$ are
  isomorphic, if they exist. (Hint: Observe that they both satisfy the universal
  property for the product of $A$ and $B$; then use Proposition 5.4.)
\end{problem}

% Problem 5.8
\begin{problem}
  \def \C {\mathsf{C}}
  Let $\C$ be a category with products. Find a reasonable candidate for the
  universal property that the product $A\times B\times C$ of three objects of $\C$
  ought to satisfy, and prove that both $(A\times B)\times C$ and $A\times
  (B\times C)$ satisfy this universal property. Deduce that $(A\times B)\times C$
  and $A\times (B\times C)$ are necessarily isomorphic.
\end{problem}

% Problem 5.10
\begin{problem}
  \def \Set {\mathsf{Set}}
  Push the envelope a little further still, and define products and coproducts
  for families (i.e., indexed sets) of objects of a category.

  Do these exist in $\Set$?

  It is common to denote the product $A\times\cdots\times A$ ($n$ times) by $A^n$.
\end{problem}

% Problem 5.11
\begin{problem}
  \newcommand{\quot}[2]{#1/\!\!#2\,\,}
  \newcommand{\quotntws}[2]{#1/\!\!#2}
  Let $A$, resp. $B$, be a set, endowed with an equivalence relation $\sim_A$,
  resp. $\sim_B$.
  Define a relation $\sim$ on $A\times B$ by setting
  \[ (a_1, b_1) \sim (a_2, b_2) \iff a_1 \sim_A a_2 \text{ and } b_1 \sim_B b_2. \]
  (This is immediately seen to be an equivalence relation.)
  \begin{itemize}
  \item Use the universal property for quotients (\S5.3) to establish that there are
  functions
  \[ \quot{(A\times B)}{\sim} \to \quotntws{A}{\sim_A},
     \quot{(A\times B)}{\sim} \to \quotntws{B}{\sim_B}. \]
  \item Prove that $\quotntws{(A\times B)}{\sim}$, with these two functions,
  satisfies the universal property for the product of $\quotntws{A}{\sim_A}$ and
  $\quotntws{B}{\sim_B}$.
  \item Conclude (without further work) that $\quot{(A\times
  B)}{\sim}\cong(\quotntws{A}{\sim_A})\times(\quotntws{B}{\sim_B}).$
  \end{itemize}
\end{problem}

% Problem 5.12
\begin{problem}
  \def \C {\mathsf{C}}
  \def \Set {\mathsf{Set}}
  Define the notions of fibered products and fibered coproducts, as terminal
  objects of the categories $\C^{\alpha,\beta}, \C_{\alpha,\beta}$ considered in
  Example 3.10 (cf. also Exercise 3.11), by stating carefully the corresponding
  universal properties.

  As it happens, $\Set$ has both fibered products and coproducts. Define these
  objects `concretely', in terms of naive yet theory. [II.2.9, III.6.10, III.6.11]
\end{problem}
